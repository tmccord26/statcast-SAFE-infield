% Options for packages loaded elsewhere
% Options for packages loaded elsewhere
\PassOptionsToPackage{unicode}{hyperref}
\PassOptionsToPackage{hyphens}{url}
\PassOptionsToPackage{dvipsnames,svgnames,x11names}{xcolor}
%
\documentclass[
]{agujournal2019}
\usepackage{xcolor}
\usepackage{amsmath,amssymb}
\setcounter{secnumdepth}{5}
\usepackage{iftex}
\ifPDFTeX
  \usepackage[T1]{fontenc}
  \usepackage[utf8]{inputenc}
  \usepackage{textcomp} % provide euro and other symbols
\else % if luatex or xetex
  \usepackage{unicode-math} % this also loads fontspec
  \defaultfontfeatures{Scale=MatchLowercase}
  \defaultfontfeatures[\rmfamily]{Ligatures=TeX,Scale=1}
\fi
\usepackage{lmodern}
\ifPDFTeX\else
  % xetex/luatex font selection
\fi
% Use upquote if available, for straight quotes in verbatim environments
\IfFileExists{upquote.sty}{\usepackage{upquote}}{}
\IfFileExists{microtype.sty}{% use microtype if available
  \usepackage[]{microtype}
  \UseMicrotypeSet[protrusion]{basicmath} % disable protrusion for tt fonts
}{}
\makeatletter
\@ifundefined{KOMAClassName}{% if non-KOMA class
  \IfFileExists{parskip.sty}{%
    \usepackage{parskip}
  }{% else
    \setlength{\parindent}{0pt}
    \setlength{\parskip}{6pt plus 2pt minus 1pt}}
}{% if KOMA class
  \KOMAoptions{parskip=half}}
\makeatother
% Make \paragraph and \subparagraph free-standing
\makeatletter
\ifx\paragraph\undefined\else
  \let\oldparagraph\paragraph
  \renewcommand{\paragraph}{
    \@ifstar
      \xxxParagraphStar
      \xxxParagraphNoStar
  }
  \newcommand{\xxxParagraphStar}[1]{\oldparagraph*{#1}\mbox{}}
  \newcommand{\xxxParagraphNoStar}[1]{\oldparagraph{#1}\mbox{}}
\fi
\ifx\subparagraph\undefined\else
  \let\oldsubparagraph\subparagraph
  \renewcommand{\subparagraph}{
    \@ifstar
      \xxxSubParagraphStar
      \xxxSubParagraphNoStar
  }
  \newcommand{\xxxSubParagraphStar}[1]{\oldsubparagraph*{#1}\mbox{}}
  \newcommand{\xxxSubParagraphNoStar}[1]{\oldsubparagraph{#1}\mbox{}}
\fi
\makeatother


\usepackage{longtable,booktabs,array}
\usepackage{calc} % for calculating minipage widths
% Correct order of tables after \paragraph or \subparagraph
\usepackage{etoolbox}
\makeatletter
\patchcmd\longtable{\par}{\if@noskipsec\mbox{}\fi\par}{}{}
\makeatother
% Allow footnotes in longtable head/foot
\IfFileExists{footnotehyper.sty}{\usepackage{footnotehyper}}{\usepackage{footnote}}
\makesavenoteenv{longtable}
\usepackage{graphicx}
\makeatletter
\newsavebox\pandoc@box
\newcommand*\pandocbounded[1]{% scales image to fit in text height/width
  \sbox\pandoc@box{#1}%
  \Gscale@div\@tempa{\textheight}{\dimexpr\ht\pandoc@box+\dp\pandoc@box\relax}%
  \Gscale@div\@tempb{\linewidth}{\wd\pandoc@box}%
  \ifdim\@tempb\p@<\@tempa\p@\let\@tempa\@tempb\fi% select the smaller of both
  \ifdim\@tempa\p@<\p@\scalebox{\@tempa}{\usebox\pandoc@box}%
  \else\usebox{\pandoc@box}%
  \fi%
}
% Set default figure placement to htbp
\def\fps@figure{htbp}
\makeatother


% definitions for citeproc citations
\NewDocumentCommand\citeproctext{}{}
\NewDocumentCommand\citeproc{mm}{%
  \begingroup\def\citeproctext{#2}\cite{#1}\endgroup}
\makeatletter
 % allow citations to break across lines
 \let\@cite@ofmt\@firstofone
 % avoid brackets around text for \cite:
 \def\@biblabel#1{}
 \def\@cite#1#2{{#1\if@tempswa , #2\fi}}
\makeatother
\newlength{\cslhangindent}
\setlength{\cslhangindent}{1.5em}
\newlength{\csllabelwidth}
\setlength{\csllabelwidth}{3em}
\newenvironment{CSLReferences}[2] % #1 hanging-indent, #2 entry-spacing
 {\begin{list}{}{%
  \setlength{\itemindent}{0pt}
  \setlength{\leftmargin}{0pt}
  \setlength{\parsep}{0pt}
  % turn on hanging indent if param 1 is 1
  \ifodd #1
   \setlength{\leftmargin}{\cslhangindent}
   \setlength{\itemindent}{-1\cslhangindent}
  \fi
  % set entry spacing
  \setlength{\itemsep}{#2\baselineskip}}}
 {\end{list}}
\usepackage{calc}
\newcommand{\CSLBlock}[1]{\hfill\break\parbox[t]{\linewidth}{\strut\ignorespaces#1\strut}}
\newcommand{\CSLLeftMargin}[1]{\parbox[t]{\csllabelwidth}{\strut#1\strut}}
\newcommand{\CSLRightInline}[1]{\parbox[t]{\linewidth - \csllabelwidth}{\strut#1\strut}}
\newcommand{\CSLIndent}[1]{\hspace{\cslhangindent}#1}



\setlength{\emergencystretch}{3em} % prevent overfull lines

\providecommand{\tightlist}{%
  \setlength{\itemsep}{0pt}\setlength{\parskip}{0pt}}



 


\usepackage{url} %this package should fix any errors with URLs in refs.
\usepackage{lineno}
\usepackage[inline]{trackchanges} %for better track changes. finalnew option will compile document with changes incorporated.
\usepackage{soul}
\linenumbers
\makeatletter
\@ifpackageloaded{caption}{}{\usepackage{caption}}
\AtBeginDocument{%
\ifdefined\contentsname
  \renewcommand*\contentsname{Table of contents}
\else
  \newcommand\contentsname{Table of contents}
\fi
\ifdefined\listfigurename
  \renewcommand*\listfigurename{List of Figures}
\else
  \newcommand\listfigurename{List of Figures}
\fi
\ifdefined\listtablename
  \renewcommand*\listtablename{List of Tables}
\else
  \newcommand\listtablename{List of Tables}
\fi
\ifdefined\figurename
  \renewcommand*\figurename{Figure}
\else
  \newcommand\figurename{Figure}
\fi
\ifdefined\tablename
  \renewcommand*\tablename{Table}
\else
  \newcommand\tablename{Table}
\fi
}
\@ifpackageloaded{float}{}{\usepackage{float}}
\floatstyle{ruled}
\@ifundefined{c@chapter}{\newfloat{codelisting}{h}{lop}}{\newfloat{codelisting}{h}{lop}[chapter]}
\floatname{codelisting}{Listing}
\newcommand*\listoflistings{\listof{codelisting}{List of Listings}}
\makeatother
\makeatletter
\makeatother
\makeatletter
\@ifpackageloaded{caption}{}{\usepackage{caption}}
\@ifpackageloaded{subcaption}{}{\usepackage{subcaption}}
\makeatother
\usepackage{bookmark}
\IfFileExists{xurl.sty}{\usepackage{xurl}}{} % add URL line breaks if available
\urlstyle{same}
\hypersetup{
  pdftitle={Revisiting the SAFE Framework in the Statcast Era: A Modernized Approach to Evalutating MLB Infield Defense},
  pdfauthor={Tyler McCord},
  pdfkeywords={Bayesian modeling, Sports analytics, Baseball defense},
  colorlinks=true,
  linkcolor={blue},
  filecolor={Maroon},
  citecolor={Blue},
  urlcolor={Blue},
  pdfcreator={LaTeX via pandoc}}



\draftfalse

\begin{document}
\title{Revisiting the SAFE Framework in the Statcast Era: A Modernized
Approach to Evalutating MLB Infield Defense}

\authors{Tyler McCord\affil{1}}
\affiliation{1}{Amherst College, }
\correspondingauthor{Tyler McCord}{tmccord26@amherst.edu}


\begin{abstract}
High resolution tracking data has transformed player evaluation in Major
League Baseball (MLB), enabling high-level analysis of player
performance. While public analyses on batting and pitching have advanced
rapidly, defensive evaluation has been comparatively underdeveloped. The
SAFE (Spatially Adjusted Fielding Evaluation) framework, introduced by
Jensen et al. (2009), was the first effort in the public sphere to
evaluate defense as a continuous space. We revisit the SAFE framework
using modern Statcast data with an emphasis on infield defense, a
notable struggle for prior defensive metrics. {[}Placeholder for
results{]}
\end{abstract}





\section{Introduction}\label{introduction}

The evaluation of batting and pitching in baseball has been at the
forefront of sports analytics for decades, mostly due to their discrete
nature and the availability of relevant, quantifiable data. It is
relatively simple to measure the outcome of a plate appearance or a
pitch, making it easier to develop metrics that accurately reflect
player performance in these areas. In contrast, defensive evaluation has
lagged behind due to the continuous, spatio-temporal nature of fielding.

Still, Major League Baseball (MLB) organizations are faced with
important decisions regarding defense, such as positioning players,
making defensive substitutions, and evaluating trade-offs between
offensive and defensive abilities. At the end of each season, MLB issues
Gold Glove awards to the best defenders at each position, highlighting
the importance of defense in the game.

Before the advent of high-resolution player tracking data, teams relied
on simple defensive metrics such as fielding percentage, which
calculates the percentage of plays a fielder successfully makes, and
errors, which count the number of plays that the player does not make
that the average fielder would. However, errors are prone to
subjectivity, as they depend primarily on the official scorer's
judgement. These metrics also fail to capture the full scope of a
player's defensive contributions, as they do not account for factors
such as range, positioning, and the difficulty of plays made.

Statisticians have tried to find ways to quantify the nuances of
defense. In 2003, Mitchel Lichtman introduced the Ultimate Zone Rating
(UZR) metric, which attempted to evaluate defense by dividing the field
into discrete zones and assigning run values to plays made or not made
within those zones. This run-based approach allowed statisticians to
understand the stakes of each defensive play.

In 2009, Jensen et al. (2009) introduced the SAFE (Spatially Adjusted
Fielding Evaluation) framework, which built upon UZR by using a
hierarchical Bayesian model to evaluate defense as a continuous surface.
The SAFE framework uses estimates of player location, ball location, and
ball velocity to model the probability of a fielder making a play on a
batted ball, allowing for a more nuanced evaluation of defensive
performance. The model combines the probability of a made play with the
run consequences of that play to estimate the overall defensive
contribution of a player in terms of runs saved or allowed. The
hierarchical Bayesian structure also allows for the sharing of
information across players and positions, improving estimates for
players with limited data. However, this model is limited by the
accuracy and reliability of the underlying data used to estimate player
and ball locations. These data, provided by Baseball Info Solutions,
used hand-annotated video footage to estimate ball location and
velocity. Even then, the starting location of the fielder at a given
position was estimated by the authors by using the average location of
balls caught by that position.

Notably, the results of Jensen et al. (2009) showed that the
autocorrelation of defensive metrics from year to year was quite low for
infielders. This shortcoming suggests that the original SAFE model
performed poorly in evaluating infield defense, relative to outfield
defense.

Since the publication of the SAFE framework, MLB has introduced
Statcast, a high-resolution player tracking system that uses a
combination of radar and camera technology to track the movement of
players and the ball in real-time. Statcast provides a wealth of data
that was previously unavailable, including precise measurements of
player and ball locations, velocities, and trajectories. This data has
the potential to revolutionize defensive evaluation in baseball,
allowing for more accurate and reliable estimates of defensive
performance.

In this paper, we modernize the original SAFE framework for infielders
using three years Statcast data (2023-2025). We perform a reproduction
of the original SAFE model using the new data, and compare the validity
of these results to those of Jensen et al. (2009). We also pose an
improved model, with additional covariates that were not available in
the original SAFE framework.

\section{Data}\label{data}

Our evaluation of infield defense is based on Statcast data from
2023-2025. Although Statcast data has been publicly available since
2015, we focus on the most recent three years because the infield
``shift'', a defensive strategy where infielders position themselves in
extreme positions based on the batter's hitting tendencies, was banned
following the 2022 season. We believe that narrowing the frame of our
analysis to non-shifted seasons will yield more accurate estimates due
to more consistent estimates for fielder locations. We obtain data on
batted balls through the \texttt{baseballr} package in R, which provides
a convenient interface for scraping Statcast data from MLB's public API.
The result is an .Rds file for each year of interest where each
observation corresponds to a single batted ball in play (BIP) event.
Further, as an extension of the original SAFE framework, we extract
information on individual player positioning before each pitch by using
the ``Fielder Positioning'' page on Baseball Savant. The location for
each infielder on a given play is not publicly available

For each batted ball in play, we extract the relevant information needed
to identify the fielder responsible for making a play, the location and
velocity of the batted ball, and the outcome of the play.

Using these data, we derive the following factors for each batted ball:

\begin{itemize}
\tightlist
\item
  \textbf{successful\_play}: A binary indicator of whether the fielder
  successfully made a play on the batted ball (1 = successful play, 0 =
  unsuccessful play). For ground balls, this is defined as whether the
  fielder was able to field the ball and record at least one out. For
  fly balls/line drives, this is defined as whether the fielder was able
  to catch the ball before it touched the ground.
\item
  \textbf{location\_x, location\_y}: The (x, y) coordinates of the
  batted ball when it reaches the fielder's location, measured in feet
  from home plate. The origin (0, 0) is at home plate, with the positive
  x-axis extending towards first base and the positive y-axis extending
  towards second base.
\item
  \textbf{spray\_angle}: Derived from the (x, y) coordinates, this angle
  represents the direction of the batted ball relative to home plate,
  measured in degrees. The first base foul line represents 45 degrees,
  second base is 0 degrees, and the third base foul line is -45 degrees.
\item
  \textbf{launch\_velocity}: The velocity at which a ball is hit off the
  bat, measured in miles per hour (mph).
\item
  \textbf{out\_\{pos\}}: A binary indicator for each infielder position
  (1B, 2B, SS, 3B) indicating whether or not the player at that position
  recorded a successful play on the batted ball.
\end{itemize}

The resulting dataset contains 372,260 batted balls in play from the
2023-2025 seasons.

\section*{References}\label{references}
\addcontentsline{toc}{section}{References}

\phantomsection\label{refs}
\begin{CSLReferences}{1}{0}
\vspace{1em}

\bibitem[\citeproctext]{ref-jensen_2009}
Jensen, S. T., Shirley, K. E., \& Wyner, A. J. (2009). Bayesball: A
bayesian hierarchical model for evaluating fielding in major league
baseball. \emph{The Annals of Applied Statistics}, \emph{3}(2),
491--520. \url{https://doi.org/10.1214/08-AOAS228}

\end{CSLReferences}




\end{document}
